\documentclass[standalone]{beamer}

\begin{document}

\section{進階經典算法}

\subsection{支配樹}

\newcommand{\idom}{\operatorname{idom}}
\newcommand{\sdom}{\operatorname{sdom}}
\newcommand{\dom}{\operatorname{dom}}

\begin{frame}{\ctitle{Motivation}}
  \onslide<1-> {
    \begin{problem}[Useful Roads]
      給一個 $n$ 點 $m$ 邊的有向圖,求出所有有用的邊。一個邊 $e$ 是有用的若存在一個點 $x$ 使得有
      一條從 $1$ 到 $x$ 的簡單路徑通過 $e$。

      \begin{itemize}
        \item $2 \leq n, m \leq 2 \times 10^5$
      \end{itemize}
    \end{problem}
  }
  \onslide<2-> {
    \begin{figure}[H]
      \centering
      \begin{tikzpicture}[scale=1.5]
        \tikzset{vertex/.style = {shape=circle,draw,minimum size=0.5em}}
        \tikzset{edge/.style = {-,> = latex'}}
        \node[vertex, fill=gray] (1) at (0, 0) {};
        \node[vertex] (2) at (-1, -1) {};
        \node[vertex] (3) at (0, -1) {};
        \node[vertex] (4) at (1, -1) {};
        \node[vertex] (5) at (-0.5, -2) {};
        \node[vertex] (6) at (0.5, -2) {};

        \draw[edge,->] (1) to (2);
        \draw[edge,->] (1) to (3);
        \draw[edge,->] (1) to (4);
        \draw[edge,->] (4) to (3);
        \draw[edge,->] (3) to (2);
        \draw[edge,->] (3) to (5);
        \draw[edge,->] (5) to (6);
        \draw[edge,->] (6) to (3);
        \draw[edge,->] (6) to (4);
      \end{tikzpicture}
    \end{figure}
  }
\end{frame}

\begin{frame}{\ctitle{Definition}}
  \onslide<1-> {
    \begin{definition}
      給定一個有向圖 $G$ 以及源點 $r$,定義節點 $d$ \textbf{支配(dominate)}節點 $x$ 若所有
      $r$ 到 $x$ 的路徑都必須經過 $d$。若 $d \neq x$,則稱 $d$ 嚴格支配 $x$。
    \end{definition}
  }
  \onslide<2-> {
    \begin{definition}
      對於節點 $x \neq r$,$x$ 的\textbf{直接支配者(Immediate Dominator)} $\idom(x)$ 滿
      足

      \begin{itemize}
        \item $\idom(x)$ 支配 $x$ 且 $\idom(x) \neq x$,且
        \item 對於所有支配 $x$ 的 $y \neq x$,$y$ 也支配 $\idom(x)$。
      \end{itemize}
    \end{definition}
  }
\end{frame}

\begin{frame}{\ctitle{Example}}
  \begin{figure}[H]
    \centering
    \begin{tikzpicture}[scale=1.5]
      \tikzset{vertex/.style = {shape=circle,draw,minimum size=0.5em}}
      \tikzset{edge/.style = {-,> = latex'}}
      \node[vertex] (r) at (0, 0) {\tiny 1};
      \node[vertex] (b) at (-1, -1) {\tiny 8};

      \node[vertex] (a) at (-1.8, -1.5) {\tiny{11}};
      \node[vertex] (d) at (-1.8, -2) {\tiny{12}};
      \node[vertex] (l) at (-1.8, -2.5) {\tiny{13}};
      \node[vertex] (h) at (-1, -3) {\tiny{10}};
      \node[vertex] (e) at (-0.5, -2) {\tiny 9};

      \node[vertex] (c) at (1, -1) {\tiny 2};
      \node[vertex] (f) at (0.5, -2) {\tiny 3};
      \node[vertex] (i) at (1, -3) {\tiny 4};
      \node[vertex] (g) at (1.8, -1.8) {\tiny 6};
      \node[vertex] (j) at (1.8, -2.5) {\tiny 7};

      \node[vertex] (k) at (0, -4) {\tiny 5};

      \draw[edge,->] (r) to (b);
      \draw[edge,->] (r) to (c);
      \draw[edge,->,bend right=30] (r) to (a);
      \draw[edge,->] (b) to (a);
      \draw[edge,->] (b) to (e);
      \draw[edge,->] (b) to (d);
      \draw[edge,->,bend left=30] (e) to (h);
      \draw[edge,->,bend left=30] (h) to (e);
      \draw[edge,->] (a) to (d);
      \draw[edge,->] (d) to (l);
      \draw[edge,->] (l) to (h);
      \draw[edge,->] (h) to (k);

      \draw[edge,->] (c) to (f);
      \draw[edge,->] (c) to (g);
      \draw[edge,->] (f) to (i);
      \draw[edge,->] (g) to (j);
      \draw[edge,->] (g) to (i);
      \draw[edge,->] (j) to (i);

      \draw[edge,->] (k) to (r);
      \draw[edge,->,bend left=30] (k) to (i);
      \draw[edge,->,bend left=30] (i) to (k);
    \end{tikzpicture}
  \end{figure}
\end{frame}

\begin{frame}{\ctitle{Semi-Dominator}}
  \begin{columns}
    \begin{column}{0.45\textwidth}
      \onslide<1-> {
        \begin{definition}
          對於 $x \neq r$,$x$ 的半支配者 $\sdom(x)$ 為所有 $y$ 使得存在一條
          $y, v_1, v_2, \ldots, v_k, x$ 且所有 $v_i$ 都大於 $x$ 的路徑中,最小的 $y$。
        \end{definition}
      }
    \end{column}
    \begin{column}{0.55\textwidth}
      \onslide<2-> {
        \begin{figure}[H]
          \centering
          \begin{tikzpicture}[scale=1.2]
            \tikzset{vertex/.style = {shape=circle,draw,minimum size=1em}}
            \tikzset{edge/.style = {-,> = latex'}}

            \node[vertex] (r) at (0, -1) {\tiny 1};

            \node[vertex] (c) at (-1, -2) {\tiny 2};
            \node[vertex] (f) at (-1.5, -3) {\tiny 3};
            \node[vertex] (i) at (-1.5, -4) {\tiny 4};
            \node[vertex] (k) at (-1.5, -5) {\tiny 5};
            \node[vertex] (g) at (-0.5, -3) {\tiny 6};
            \node[vertex] (j) at (-0.5, -4) {\tiny 7};

            \node[vertex] (b) at (1, -2) {\tiny 8};
            \node[vertex] (e) at (0.5, -3) {\tiny 9};
            \node[vertex] (h) at (0.5, -4) {\tiny{10}};
            \node[vertex] (a) at (1.5, -3) {\tiny{11}};
            \node[vertex] (d) at (1.5, -4) {\tiny{12}};
            \node[vertex] (l) at (1.5, -5) {\tiny{13}};

            \draw[edge,->] (r) to (b);
            \draw[edge,->] (r) to (c);
            \draw[edge,->,bend left=50,dashed] (r) to (a);
            \draw[edge,->] (b) to (a);
            \draw[edge,->] (b) to (e);
            \draw[edge,->,dashed,bend right] (b) to (d);
            \draw[edge,->,bend left=30] (e) to (h);
            \draw[edge,->,bend left=30,dashed] (h) to (e);
            \draw[edge,->] (a) to (d);
            \draw[edge,->] (d) to (l);
            \draw[edge,->,dashed] (l) to (h);
            \draw[edge,->,dashed] (h) to (k);

            \draw[edge,->] (c) to (f);
            \draw[edge,->] (c) to (g);
            \draw[edge,->] (f) to (i);
            \draw[edge,->,dashed] (g) to (j);
            \draw[edge,->] (g) to (i);
            \draw[edge,->,dashed] (j) to (i);

            \draw[edge,->,bend left=90,dashed] (k) to (r);
            \draw[edge,->,bend left=30,dashed] (k) to (i);
            \draw[edge,->,bend left=30] (i) to (k);
          \end{tikzpicture}
        \end{figure}
      }
    \end{column}
  \end{columns}
\end{frame}

\begin{frame}{\ctitle{Lemmas}}
  \onslide<1-> {
    \begin{lemma}
      對於 $T$ 上的兩個節點 $x < y$,任何 $x$ 到 $y$ 的路徑都會包含 $x$ 與 $y$ 在 $T$ 上的一個共同祖先。
    \end{lemma}
  }
  \onslide<2-> {
    \begin{lemma}
      對於 $x \neq r$,$\sdom(x)$ 是 $x$ 在 $T$ 上的祖先且 $\sdom(x) \neq x$。
    \end{lemma}
  }
  \onslide<3-> {
    \begin{lemma}
      對於 $x \neq r$,$\idom(x)$ 是 $\sdom(x)$ 在 $T$ 上的祖先。
    \end{lemma}
  }
\end{frame}

\begin{frame}{\ctitle{Lemmas}}
  \onslide<1-> {
    \begin{lemma}
      對於 $x \neq r$ 以及 $y \neq r$,若 $x$ 是 $y$ 在 $T$ 上的祖先,則 (1) $x$ 是 $\idom(y)$ 的祖先或 (2) $\idom(y)$ 是 $\idom(x)$ 的祖先。
    \end{lemma}
  }
  \onslide<2-> {
    \begin{lemma}
      對於 $x \neq r$,如果對於所有在 $\sdom(x)$ 與 $x$ 於 $T$ 上的路徑中的節點
      $y \neq \sdom(x)$,都有 $\sdom(y) \geq \sdom(x)$ 的話,則 $\idom(x) = \sdom(x)$。
    \end{lemma}
  }
  \onslide<3-> {
    \begin{lemma}
      對於 $x \neq r$,令 $y$ 為 $\sdom(x)$ 與 $x$ 於 $T$ 上路徑中,有最小 $\sdom(y)$ 的節
      點。則 $\sdom(y) \leq \sdom(x)$ 且 $\idom(x) = \idom(y)$。
    \end{lemma}
  }
\end{frame}

\begin{frame}{\ctitle{Algorithm}}
  \begin{itemize}
    \item 支配者:
      \onslide<2-> {
        \[
          \idom(x) =
            \begin{cases}
              \sdom(x), & \sdom(x) = \sdom(y) \\
              \idom(y), & \sdom(x) \neq \sdom(y) \\
            \end{cases}
        \]
      }
    \item 半支配者:
      \onslide<3-> {
        \[
          \sdom(x) = \min
            \begin{cases}
              y, & (y, x) \in E(G), y < x \\
              \sdom(z), & (y, x) \in E(G), z > x \text{ 是 $y$ 的祖先} \\
            \end{cases}
        \]
      }
  \end{itemize}
\end{frame}

\begin{frame}{\ctitle{Algorithm}}
  \begin{algorithm}[H]
    \KwData{有向圖 $G = (V, E)$、根 $r \in V$}

    從 $r$ 跑一次 DFS 得到 DFS 序列 $v_1, v_2, \ldots, v_n$ \\

    \onslide<2-> {
      \For{$i = n$ \KwTo $1$} {
        $\sdom(v_i) \gets \min\{\sdom(\operatorname{Eval}(u)) \mid (u, v_i) \in E\}$ \\

        \onslide<3-> {
          \For{$u \in \sdom^{-1}(v_i)$} {
            $p \gets \operatorname{Eval}(u)$ \\
            $\dom(u) =
              \begin{cases}
                v_i, & \sdom(p) = x \\
                p, & \sdom(p) \neq x \\
              \end{cases}$ \\
          }
        }

        \onslide<4-> {
          $\operatorname{Link}(v_i, \operatorname{par}(v_i))$
        }
      }
    }

    \onslide<5-> {
      \For{$i = 1$ \KwTo $n$} {
        $\idom(v_i) =
          \begin{cases}
            \dom(v_i), & \sdom(v_i) = \dom(v_i) \\
            \dom(\dom(v_i)), & \sdom(v_i) \neq \dom(v_i) \\
          \end{cases}$
      }
    }
  \end{algorithm}
\end{frame}

\subsection{有向最小生成樹}

\begin{frame}{\ctitle{Motivation}}
  \begin{problem}[GGS-DDU]
    有 $n$ 道關卡,第 $i$ 道有 $0, 1, \ldots, A_i$、共 $A_i + 1$ 個等級。一開始你在每個關
    卡都是 $0$ 等。有 $m$ 個道具,對於第 $i$ 個道具,假如你在第 $c_i$ 關達到至少 $L_{1,i}$
    等的話,那你可以花 $p_i$ 元在第 $d_i$ 關晉升到 $L_{2,i}$ 等。請問要在每一關都達到滿等最少
    需要花多少錢?

    \begin{itemize}
      \item $1 \leq n \leq 50$
      \item $0 \leq m \leq 2000$
      \item $A_i$ 總和不超過 $500$
    \end{itemize}
  \end{problem}
\end{frame}

\begin{frame}{\ctitle{Idea}}
  \begin{figure}[H]
    \centering
    \begin{tikzpicture}[scale=1.5]
      \tikzset{vertex/.style = {shape=circle,draw,minimum size=1em}}
      \tikzset{edge/.style = {-,> = latex'}}
      \node[vertex, fill=gray] (r) at (0, 0) {};
      \node[vertex] (v1) at (1, -1) {};
      \node[vertex] (v2) at (1, -2) {};
      \node[vertex] (v3) at (0, -2) {};
      \node[vertex] (x) at (-1, -1) {};
      \node[vertex] (v4) at (-1, -2) {};
      \node[vertex] (v5) at (-1, -3) {};
      \node[vertex] (v6) at (1, -3) {};

      \only<1> {
        \draw[edge,->] (r) -- (v1) node [midway,right] {$1$};
        \draw[edge,->] (v1) -- (v2) node [midway,left] {$1$};
        \draw[edge,->] (v2) -- (v3) node [midway,below] {$1$};
        \draw[edge,->] (r) -- (x) node [midway,left] {$1$};
        \draw[edge,->] (x) -- (v4) node [midway,left] {$1$};
        \draw[edge,->] (v4) -- (v5) node [midway,left] {$1$};
        \draw[edge,->] (v5) -- (v6) node [midway,above] {$2$};
        \draw[edge,->] (v2) to node [midway,left] {$1$} (v6) ;
        \draw[edge,->] (v3) -- (v4) node [midway,below] {$2$};
        \draw[edge,->,bend right] (v6) to node [midway,right] {$2$} (v1);
      }
      \only<2> {
        \draw[edge,->] (r) -- (v1) node [midway,right] {$2$};
        \draw[edge,->] (v1) -- (v2) node [midway,left] {$1$};
        \draw[edge,->] (v2) -- (v3) node [midway,below] {$1$};
        \draw[edge,->] (r) -- (x) node [midway,left] {$1$};
        \draw[edge,->] (x) -- (v4) node [midway,left] {$2$};
        \draw[edge,->] (v4) -- (v5) node [midway,left] {$1$};
        \draw[edge,->] (v5) -- (v6) node [midway,above] {$1$};
        \draw[edge,->] (v2) -- (v6) node [midway,left] {$2$};
        \draw[edge,->] (v3) -- (v4) node [midway,below] {$1$};
        \draw[edge,->,bend right] (v6) to node [midway,right] {$1$} (v1);
      }
    \end{tikzpicture}
  \end{figure}
\end{frame}

\begin{frame}{\ctitle{Key Lemma}}
  \begin{columns}
    \begin{column}{0.5\textwidth}
      \onslide<1-> {
        \begin{lemma}
          若對於除了 $r$ 以外的點都蒐集它們最小的入邊,而這 $n - 1$ 條邊形成一個環 $C$
          的話,那存在一個最小有向生成樹包含 $C$ 中的 $|C| - 1$ 條邊。
        \end{lemma}
      }
    \end{column}
    \begin{column}{0.5\textwidth}
      \onslide<2-> {
        \begin{figure}[H]
          \centering
          \begin{tikzpicture}[scale=1.5]
            \tikzset{vertex/.style = {shape=circle,draw,minimum size=1em}}
            \tikzset{edge/.style = {-,> = latex'}}
            \node[vertex, fill=gray] (r) at (0, 0) {};
            \node[vertex] (v1) at (1, -1) {};
            \node[vertex] (v2) at (1, -2) {};
            \node[vertex] (v3) at (1, -3) {};
            \node[vertex] (x) at (-1, -1) {};
            \node[vertex] (v4) at (-1, -2) {};
            \node[vertex] (v5) at (-1, -3) {};
            \node[vertex] (v6) at (0, -3.5) {};
            \draw[edge,->] (r) -- (v1);
            \draw[edge,->] (v1) -- (v2);
            \draw[edge,->] (v2) -- (v3);
            \draw[edge,->] (r) -- (x);
            \draw[edge,->] (x) -- (v4);
            \draw[edge,->] (v4) -- (v5);
            \draw[edge,->] (v5) -- (v6);
            \draw[edge,->] (v2) -- (v6);
            \draw[edge,->] (v3) -- (v4);
            \draw[edge,->] (v6) -- (v1);
          \end{tikzpicture}
        \end{figure}
      }
    \end{column}
  \end{columns}
\end{frame}

\begin{frame}{\ctitle{Algorithm}}
  \begin{algorithm}[H]
    \KwData{帶權有向圖 $G = (V, E, w)$、根 $r \in V$}
    \KwResult{$G$ 的有向最小生成樹的權值}

    \onslide<2-> {
      $in(v) \gets$ $v$ 的最小權入邊($v \neq r$) \\
      $G_{in} \gets$ $in(v)$ 所構成的子圖 \\
    }
    \onslide<3-> {
      \If{$G_{in}$ 沒有有向環} {
        \Return $\sum_{v \neq r}w(in(v))$
      }
    }
    \onslide<4-> {
      $C \gets$ $G_{in}$ 中的一個有向環 \\
      $V^\prime \gets (V \setminus V(C)) \cup \{c\}$ \\
      $E^\prime \gets \{(f(u), f(v)) \mid (u, v) \in E\}$ \\
    }
    \onslide<5-> {
      $w^\prime((f(u), f(v))) \gets
        \begin{cases}
          w((u, v)), & v \not\in V(C) \\
          w((u, v)) - w(in(v)) & v \in V(C) \\
        \end{cases}$ \\
    }
    \onslide<6-> {
      \Return $\operatorname{DMST}(G^\prime = (V^\prime, E^\prime, w^\prime), r) + \sum_{v \neq r}w(in(v))$
    }
  \end{algorithm}
\end{frame}

\begin{frame}{\ctitle{Subquadratic Implementation}}
  \begin{tabular}{| c | c | c | c | c |}
    \hline
    & \texttt{top()} & \texttt{pop()} & \texttt{push()} & \texttt{merge()} \\\hline
    Leftist Tree & $O(1)$ & $O(\log{n})$ & $O(\log{n})$ & $O(\log{n})$ \\\hline
    Fibonacci Heap & $O(1)$ & $O(\log{n})$ & $O(1)$ & $O(1)$ \\\hline
    Pairing Heap & $O(1)$ & $O(\log{n})$ & $O(1)$ & $O(1)$ \\\hline
  \end{tabular}

  \begin{itemize}
    \item Library Checker: \url{https://judge.yosupo.jp/problem/directedmst}
  \end{itemize}
\end{frame}

\end{document}
